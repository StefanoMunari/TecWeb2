\section{Valutazione}
Nel presente documento:
\begin{itemize}
	\item per ogni sottosezione di terzo livello (i.e. 2.2.1) 
	è stata data una valutazione in una scala da 0 (inclassificabile) a 10 (ottimo);
	\item per ogni sottosezione di secondo livello (i.e. 2.1) è stata calcolata
	la media aritmetica dei risultati ottenuti dalle relative sottosezioni di livello superiore a 3
	(come si può notare dalle voci presenti in tabella);
	\item la valutazione finale è composta dalla media pesata 
	(secondo importanza) dei risultati ottenuti dalle sottosezioni di secondo 
	livello.
\end{itemize}

\begin{longtable}{| p{5cm} | p{4cm} | l |}

\hline
\hline
\textbf{Aspetto} & \textbf{Valutazione} & \textbf{Peso} \\ 
\hline
\hline

\nameref{sezioni} & $4.5$ & $0.1$ \\%4.5
\hline
\nameref{general} & $6.63$ & $0.1$ \\%(9.8+6+9+0+5+10)/6 = 6.63
\hline
\nameref{where} & $8$ & $0.09$ \\%8
\hline
\nameref{who} & $8$ & $0.09$ \\%8
\hline
\nameref{why} & $8$ & $0.09$ \\%8
\hline
\nameref{what} & $6$ & $0.09$ \\%6
\hline
\nameref{when} & $5$ & $0.09$ \\%5
\hline
\nameref{how} & $3$ & $0.09$ \\%3
\hline
\nameref{usecase} & $8.65$ & $0.11$ \\%(7.5+9.8)/2 = 8.65
\hline
\nameref{searchfun} & $8.75$ & $0.07$ \\%(7.5+10)/2 = 8.75
\hline
\nameref{contenuto} &  $8$ & $0.08$ \\%(8.5+5.5+10)/3 = 8
\hline
\hline
\textbf{Totale} & \textbf{Valutazione$*$Peso} & \textbf{6.12} \\%((0.45+0.663+0.72+0.72+0.72+0.54+0.45+0.27+0.95+0.61+0.64)*10)/11
\hline
\hline
%
% 4.5*0.1
% 6.63*0.1
% 8*0.09
% 8*0.09
% 8*0.09
% 6*0.09
% 5*0.09
% 3*0.09
% 8.65*0.11
% 8.75*0.07
% 8*0.08
%
\end{longtable}
