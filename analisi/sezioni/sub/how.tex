\subsection{How}\label{how}
\begin{center}

\textit{Come faccio ad arrivare alle sezioni principali?}

\end{center}
\begin{flushleft}
Il menù di navigazione non permette di muoversi agevolmente
in tutte le parti del sito in quanto molte sottosezioni sono \quotes{nascoste}
nell'utilizzo dei pulsanti (come spiegato in §\ref{sezioni}) compromettendo quasi totalmente la navigazione. \\
Risulta oneroso
consultare e cercare le informazioni di cui l'utente necessita.\\
Una possibile soluzione consiste in:
\begin{itemize}
	\item ridurre le molte pagine di scroll, presenti nella maggior parte
delle sezioni, e creare delle sezioni 
a profondità maggiore rendendo l'attuale menù a tendina (con due livelli
 di profondità sarebbe sufficiente);
\item dare al menù una posizione
fissa rispetto al browser in modo che, anche nel caso si utilizzasse lo scrolling
verticale, sia sempre possibile per l'utente avere le informazioni relative all'asse
\textit{How}.
\end{itemize}
    \begin{figure}[ht]
    \centering
    \includegraphics[scale=0.38,keepaspectratio]{{figure/3/how0}.png}
    \caption{Homepage - Menù ad un solo livello}
    \end{figure}
    \FloatBarrier 
Risultato : \textit{3}
\end{flushleft}