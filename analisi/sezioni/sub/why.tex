\subsection{Why}
\begin{center}

\textit{Quali benefici offre il sito e quali sono le motivazioni per visitarlo?}

\end{center}
\begin{flushleft}
Come blurb al motto presente in Homepage viene spiegato quali benefici il sito 
possa offrire: \\
\quotes{an invaluable source for competitive intelligence and strategic insight}
\\
Per rendere più \quotes{umano} l'asse \quotes{Why}, e quindi convincere l'utente,
viene messo come sfondo della Homepage un CEO sorridente che utilizza i servizi di Alexa
per la sua azienda. A rimarcare ciò contribuisce la citazione sottostante al 
blurb appena descritto. Infine (sempre nella prima pagina)
vengono citate alcune aziende clienti di Alexa. Un'osservazione riguarda il contenuto
che, in questo caso, viene presentato in stile \quotes{slogan} e con poche informazioni
teniche e specifiche. Quest'ultima non è una buona prassi soprattutto perchè il target di utenza
per questa tipologia di siti è definito in partenza (vedi \ref{scopo}) e quindi
molto probabilmente gli utenti si aspettano delle informazioni più specifiche 
e possono rimanere infastiditi dall'uso eccessivo di slogan.

\end{flushleft}