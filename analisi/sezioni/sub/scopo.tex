\subsection{Scopo del sito}\label{scopo}
È stato analizzato il sito web \textit{Alexa} (\url{http://www.alexa.com/}). \\
Il sito consiste nella presentazione dei servizi dell'azienda Alexa Internet Inc. \\
L'azienda statunitense (sussidiaria di \textit{Amazon.com}) fornisce 
strumenti per valutare il traffico web. L'algoritmo di valutazione utilizzato è chiamato \textit{Alexa Rank}: \\
\textit{valuta il traffico web (solo domini .com) basandosi sulle informazioni di navigazione ricevute dagli utenti che 
hanno installato Alexa Toolbar, queste informazioni si basano sullo 
storico degli ultimi 3 mesi di traffico analizzato.
La prima posizione in Alexa Rank corrisponde al sito più popolare, 
l'azienda afferma: non abbiamo abbastanza dati per fare delle classifiche 
significative per siti con ranking (posizione) superiore a 100.000.} \\
Da questa breve descrizione si comprende come \textit{Alexa Rank} offra dei risultati approssimati e poco attendibili sul traffico totale del web, tuttavia può 
essere una fonte ulteriore di informazioni riguardo il trend di visita attuale dei maggiori siti.\\ 
Il target di utenza tipico consiste di blogger e webmaster, esperti di SEO, 
aziende.\\
\paragraph{Nota}
 A causa dell'ampiezza del sito web analizzato si è preferito tralasciare l'analisi \quotes{commerciale} del sito (riguardante gli aspetti
legati ai prezzi ed alla vendita dei servizi come prodotti) in quanto 
avrebbe ecceduto le ore stimate per la realizzazione dell'attività di progetto.
