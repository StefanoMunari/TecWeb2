\subsubsection{Scrolling}\label{scrolling}
Nelle sezioni principali del sito (Homepage, Plans and Pricing, Tools, Blog) 
\textbf{lo scrolling} (con display 13" 1280x800) \textbf{occupa in media} uno spazio di 
\textbf{6.5 schermi}. La cosa più preoccupante è il picco di 7.5 schermi di 
scrolling verticale raggiunto nella homepage, la pagina più importante del sito. \\Considerando che in uno scenario mobile questo tipo di scrolling
risulta accettabile, in uno scenario desktop un utente in media scrolla 2.3 schermi
quindi un tale numero di pagine da scrollare non risulta accettabile. Non a caso una delle prime impressioni che si ha nelle 
navigazione del sito è che ci siano \textit{veramente troppe informazioni} 
nelle pagine principali e che risultino quindi \textit{pesanti da navigare}. \\
Altra nota \quotes{bizzarra}: la freccia che permette di ritornare al top della 
pagina è presente solo nello scrolling della sezione \textit{Plans and Pricing},
questo può essere sintomo di un design approssimativo o di modifiche al codice
disordinate che non sono state applicate trasversalmente a tutte le pagine. \\
In conclusione lo scrolling risulta uno dei punti deboli più evidenti
del sito. \\
Risultato : \textbf{0}
