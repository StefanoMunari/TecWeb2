\subsubsection{Testo}
Il testo è in font sans-serif, presenta una dimensione leggibile (anche se non è possibile customizzarla) ed ha quasi sempre una buon contrasto rispetto ai colori di sfondo. Il testo risulta disposto orizzontalmente (non per colonne) in brevi paragrafi, in alcuni casi la disposizione dei contenuti è in forma di lista ed in rari casi 
in forma di liste affiancate. Per quest'ultima opzione le informazioni 
dovrebbero essere più strutturate (i.e. brevi sottosezioni come accennato in
\ref{overview}) altrimenti l'utente può trovare difficoltà nella ricerca all'interno
 di una struttura così complicata. 
\begin{figure}[ht]
\centering
\includegraphics[scale=0.30,keepaspectratio]{{figure/4/blurb0}.png}
\caption{Esempio di titolo con blurb}
\end{figure}
\FloatBarrier 
Ad ogni titolo di paragrafo è associato
 un breve blurb per fornire le informazioni chiave che questo contiene.\\
Risultato : \textit{8.5}