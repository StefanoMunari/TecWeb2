\subsubsection{Navigazione}
Lo spostamento interno al sito avviene attraverso un menù di navigazione posto
orizzontalmente nel top della pagina. Fortunatamente non è presente alcuna
splash page e la navigazione può iniziare senza ulteriori impedimenti 
una volta che l'utente approda nel sito. 
La navigazione tra le sezioni principali
risulta abbastanza agevole in quanto le aree cliccabili sono abbastanza grandi
ma non vengono evidenziate in modo efficace (aumenta di pochissimo l'opacità),
altra nota negativa è data dal 
menù a posizione relativa che scompare quando si scrolla oltre la prima pagina.
Si riscontra il classico problema \textit{lost in navigation} (perdita dell'asse
where) in quanto non è
presente una breadcrumb che indichi il percorso seguito dall'utente e i link che 
sono già stati visitati non vengono evidenziati con un colore diverso. I link 
sono però evidenziati con un colore azzurro e quando ci si passa sopra risultano
sottolineati. \\
Risultato : \textit{5}