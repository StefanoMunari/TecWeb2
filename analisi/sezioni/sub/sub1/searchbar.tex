\subsubsection{Barra di ricerca}
La barra di ricerca presente in un qualsiasi sito web dovrebbe permettere
all'utente di raggiungere le sezioni o i contenuti che vuole ricercare, diminuendo gli sforzi di
navigazione dati dal menù: rappresenta quindi 
l'asse \quotes{How} per le pagine interne. \\
Tipicamente la funzionalità di ricerca, se presente nel sito, viene utilizzata
dal 100\% degli utenti. \\
\paragraph{Interfaccia}
L'interfaccia offerta dalla barra di ricerca (box testuale + pulsante find) 
è in linea con le convenzioni del web.
\paragraph{Ricerca}
La ricerca effettuata dal sito non risulta essere una normale ricerca nelle pagine
interne bensì una duplicazione della funzionalità di ricerca descritta nella sezione
§\ref{overview} violando quindi una convenzione web. 
Probabilmente i creatori del sito assumono che chi lo visita 
sappia già di che sito si tratta e che cosa esso offre. Questa assunzione 
è in parte ragionevole, anche se non giustifica del tutto una funzionalità
di ricerca così particolare. Infatti, chi già conosce il sito, potrebbe preferire
la navigazione verso qualche sottosezione specifica senza dover navigare alla 
sezione padre per poi dover scrollare la pagina fino ad arrivare al pulsante
che la contiene. 
\paragraph{Box testuale}
Considerando lo scopo di questo box testuale (contenere indirizzi web) notiamo
come la lunghezza massima di testo inseribile sia pari a 21 caratteri.
\begin{figure}[ht]
\centering
\includegraphics[scale=0.35,keepaspectratio]{{figure/4/searchBox0}.png}
\caption{Search Box - particolare}
\end{figure}
\FloatBarrier
Da \href{http://datagenetics.com/blog/march22012/}{analisi sui nomi di dominio .com}
(gli unici domini analizzati da Alexa)
risulta una lunghezza media di 13.539 caratteri con una lunghezza massima di 35.
Quindi la lunghezza del box offerto da Alexa è accettabile anche se non ottimale.\\
Risultato : \textit{7.5} 