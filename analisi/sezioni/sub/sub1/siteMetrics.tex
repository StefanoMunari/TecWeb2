\subsubsection{Certified Site Metrics}\label{metrics}
Si fa riferimento alla pagina (Certified Site Metrics > Take a Tour): \\
\url{http://www.alexa.com/tools#on-site-intelligence}
\\
Screenshot disponibile al path : \textit{figure/4/siteMetrics0.png} \\ 
Questa pagina è contenuta all'interno di un pop-up.
In questo caso la scelta del pop-up è da ritenersi sensata perchè riesce a 
mitigare i problemi relativi alla navigazione del sito evidenziati in 
§\ref{scrolling}: l'utente ottiene subito l'informazione senza
bisogno di ulteriore scrolling o di attendere il caricamento della pagina.\\
Per scorrere i contenuti internamente alla pagina viene usata una lista
 orizzontale composta da più punti, sono presenti e sempre
visualizzabili le classiche frecce per scorrere la lista avanti e indietro. \\
La navigazione della pagina risulta immediata e semplice, anche se poco
convenzionale. \\
La pagina presenta in modo ottimale i contenuti che vengono spiegati in 
dettaglio (i.e. viene spiegato come vengono calcolate le metriche sul
nostro sito, perchè utilizzare un determinato strumento etc.). I contenuti vengono
forniti concisamente e spesso attraverso delle immagini informative e
semplici: questo permette all'utente di velocizzare il processo di assimilazione
dell'informazione anche se, in generale, non è una buona pratica 
preferire le immagini al testo. L'unica critica che si può sollevare riguarda
alcune immagini il cui testo non è del tutto leggibile, quindi si dovrebbero
poter ingrandire per garantire all'utente un'esperienza migliore.
 \\
Risultato : \textit{9.8}