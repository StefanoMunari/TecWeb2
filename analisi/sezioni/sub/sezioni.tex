\subsection{Sezioni}
Il sito risulta ampio e particolarmente denso di informazioni. Le sezioni in cui
le informazioni sono distribuite sono, nell'ordine:
\begin{itemize}
	\item Home;
	\item Plans and Pricing;
	\item Tools;
	\item Dashboard;
	\item Toolbar;
	\item About us;
	\item Support;
	\item Blog.
\end{itemize}
Ad una prima occhiata il sito sembra avere un solo livello di profondità ma,
analizzando attentamente ogni sezione, si nota come alcune (i.e. About us) 
presentino una colonna a sinistra con delle voci indicanti altre sezioni. \\
Screenshot disponibile al path : \textit{Figure/2/sez0.png} \\ 
Mentre alcune sezioni hanno \quotes{evitato} del tutto altri livelli di 
profondità semplicemente disponendo tutte le informazioni verticalmente nella 
pagina (come spiegato in \ref{scrolling}). Ne risulta una gran confusione 
organizzativa data anche dal grande numero di link disseminati
in tutto il sito. Altra nota negativa data dalla sezione \textit{Support} 
che stravolge il layout standard eliminando totalmente il menù di navigazione e
presentando un'interfaccia che sembra essere totalmente estranea al sito, l'unico
modo per ritornare alla navigazione è cliccare sull'icona dell'azienda in alto
a sinistra che riporta alla homepage. \\
Screenshot disponibile al path : \textit{Figure/2/sez1.png} \\ 